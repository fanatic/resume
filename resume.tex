% Jason Parrott Resume
% Created: 10/1/07
% Modified: 8/29/21

\documentclass[letterpaper,10pt]{article}

\oddsidemargin  0.0in
\evensidemargin 0.0in
\textwidth      6.5in
\headheight     0.0in
\topmargin      -0.7in
\textheight	10.0in

\pagestyle{empty}
\raggedbottom
\raggedright

\setlength{\parindent}{0in}
\setlength{\parskip}{0in}
\setlength{\itemsep}{0in}
\setlength{\topsep}{0in}
\setlength{\tabcolsep}{0in}

% If I ever move...
\newcommand{\name}{Jason Parrott}
\newcommand{\addr}{https://github.com/fanatic}
\newcommand{\phone}{203-539-1337}
\newcommand{\email}{jason@parrott.ws}


%%%%%%%%%%%%%%%%%%%%%%%%%%%%%%%%%%%%%%%%%%%%%%%%%%%%%%%%%
% Macros

% Name
\newcommand{\bigname}[1]{
	\begin{center}\fontfamily{phv}\selectfont\Huge\scshape#1\end{center}
}

% A ressection is a main section (<H1>Section</H1>)
\newenvironment{ressection}[1]{
	\vspace{4pt}
	{\fontfamily{phv}\selectfont\Large#1}
	\begin{itemize}
	\vspace{3pt}
}{
	\end{itemize}
}

% A resitem is a simple list element in a ressection (first level)
\newcommand{\resitem}[1]{
	\vspace{-4pt}
	\item \begin{flushleft} #1 \end{flushleft}
}

% A ressubitem is a simple list element in anything but a ressection (second level)
\newcommand{\ressubitem}[1]{
	\vspace{-1pt}
	\item \begin{flushleft} #1 \end{flushleft}
}

% A resbigitem is a complex list element for stuff like jobs and education:
%  Arg 1: Name of company or university
%  Arg 2: Location
%  Arg 3: Title and/or date range
\newcommand{\resbigitem}[4]{
	\vspace{-5pt}
	\item
	\begin{tabular*}{6in}{l@{\extracolsep{\fill}}r}
		\textbf{#1} & #2 \\
		\textit{#3} & #4\\
	\end{tabular*}
}

% This is a list that comes with a resbigitem
\newenvironment{ressubsec}[4]{
	\resbigitem{#1}{#2}{#3}{#4}
	\vspace{-2pt}
	\begin{itemize}
}{
	\end{itemize}
}

% This is a simple sublist
\newenvironment{reslist}[1]{
	\resitem{\textbf{#1}}
	\vspace{-5pt}
	\begin{itemize}
}{
	\end{itemize}
}

% A resitem is a simple list element in a ressection (first level)
\newcommand{\reslongitem}[2]{
	\vspace{-4pt}
	\item
	\begin{tabular*}{6in}{l@{\extracolsep{\fill}}r}
		#1 & #2 \\
	\end{tabular*}
}

% Done Macros
%%%%%%%%%%%%%%%%%%%%%%%%%%%%%%%%%%%%%%%%%%%%%%%%%%%%%%%%%

\begin{document}


% Name with horizontal rule
\bigname{\name}

\vspace{-8pt} \rule{\textwidth}{1pt}

\vspace{-1pt} {\small \addr \hfill \itshape\phone; \email}

\vspace{8 pt}

\begin{ressection}{Background}
	\item[]
	20+ year experienced platform engineer building and maintaining reliable, elegant, and supportable systems.  Specializes in systems engineering with a strong emphasis on automation, architecture, and performance optimization.
\end{ressection}


\begin{ressection}{Professional Experience}
	\begin{ressubsec}{VP, Principal Software Architect}{September 2018 - Present}{FactSet Research Systems}{Norwalk, CT}
		\ressubitem{Managed a team of seven engineers, responsible for recruiting, team building, product development, project management, technical support, training, and technical architecture}
		\ressubitem{Concieved and developed Storm, an internal load testing platform for FactSet.  Storm allows development teams to test their applications at scale, simulating millions of users, and identifying performance bottlenecks before they reach production.  Storm is a multi-cloud, multi-region, multi-protocol, and multi-tenant platform that can be used for a variety of testing scenarios (\textit{Go}, \textit{K6}, \textit{VictoriaMetrics})}
		\ressubitem{Designed, developed, and maintained Rollout, an internal feature flag management and data-driven experimentation platform for FactSet.  Rollout rapidly changed engineering culture, allowing development teams to ship features faster and more reliably to customers.  It enabled a/b testing, user events, and performance and conversion metrics to be collected and analyzed in real-time.  Scheduled changes, approval gates, and metrics-driven kill switches support company initatives for visibility and control (\textit{Go}, \textit{React}, \textit{PostgreSQL}, \textit{Kafka}, \textit{Elasticsearch})}
		\ressubitem{Inherited the CI/CD team, responsible for maintaining, upgrading, troubleshooting, and on-call support for 15000+ content and application Git-based repositories (\textit{GitHub Enterprise Server}), testing and deployment pipelines (\textit{Cloudbees}, \textit{Jenkins}), and artifact repositories (\textit{JFrog Artifactory}) for all of FactSet's client-facing development teams.  Systematically transformed the team's style of working, enabling rapid development and improved stability.}
		\ressubitem{Rewrote the FactSet.io (PaaS) HTTP Router from Node.js to Go, allowing the platform to scale to nearly 50,000 requests per second (per server) and enable new features (\textit{blue/green deployments}, \textit{mTLS application identity}, \textit{SOCKS5}, \textit{HTTP/2}, \textit{gRPC}, \textit{websockets})}
		\ressubitem{Continued to scale FactSet.io (PaaS) to handle nearly 30,000 applications, 2,000 daily active users, and 40\% year-over-year growth rate while making order-of-magnitude improvements to hardware costs, developer efficiency, and operational burden (\textit{Spinnaker})}
		\ressubitem{Improved the existing containerized Database as a Serivce platform, adding zero-config at-rest and in-flight encryption, multi-tenant Kafka plans, Elasticsearch and MSSQL support}
		\ressubitem{Led several large-scale migration projects (\textit{Kerberos} $\rightarrow$ \textit{SAML} for platform authentication, \textit{X-Ray} $\rightarrow$ \textit{Lightstep} for distributed tracing, \textit{Docker} $\rightarrow$ \textit{Firecracker} and \textit{containerd} for FactSet.io container runtime)}
		\ressubitem{Built Library, a multi-tenant, pluggable documentation and collaboration platform allowing migration off dozens of dispearate platforms into a single, searchable, and discoverable location (\textit{Go}, \textit{React}, \textit{CRDT})}
		\ressubitem{Enabled network flow visibility and monitoring for FactSet.io, allowing developers to see the flow of data through their applications, services, and databases (\textit{eBPF})}
		\ressubitem{Executed an external audit of FactSet.io, identifying and resolving dozens of security and compliance improvements, documenting them in a CSA CIAQ for teams going through audits of their own}
		\ressubitem{Migrated secrets to \textit{Hashicorp Vault}, allowing our components to securely manage their access to resources using automatic rotating tokens and eliminating long-lived secrets on disk}
		\ressubitem{Developed virtual network support in FactSet.io, allowing for isolated apps and data with secured ingress and egress as well as DNS-based service discovery for clusters and microservice architectures (\textit{IPv6}, \textit{Wireguard})}
	\end{ressubsec}

\pagebreak

	\begin{ressubsec}{Principal Software Engineer}{February 2017 - September 2018}{FactSet Research Systems}{Norwalk, CT}
		\ressubitem{Managed a team of five engineers, responsible for recruiting, team building, product development, project management, technical support, training, and technical architecture}
		\ressubitem{Build an integrated Continuous Integration solution, FactSet.io CI, providing container-based, ephemeral, zero-config unit and integration testing that runs thousands of times a day}
		\ressubitem{Improved the existing containerized Database as a Serivce platform, adding multi-region, scaling to 2000+ dbs, supporting client-facing workloads, and adding \textit{Kafka} support}
		\ressubitem{Added Amazon Web Service (\textit{AWS}) support to FactSet.io, migrating 10000+ applications and their data to the cloud with no downtime or developer intervention}
		\ressubitem{Deployed Global HTTP Routing and Load Balancing to support FactSet's continued transition to a global company, reducing latency with an easy-to-use developer experience}
		\ressubitem{Created FactSet's first Developer Portal (developer.factset.com) CRM based on \textit{Hugo}, then later \textit{Drupal} which kicked off FactSet's Digital Transformation and API initatives (now one of FactSet's fastest growing products)}
		\ressubitem{Developed a ChatOps Bot for \textit{Microsoft Teams} and \textit{Slack} to allow FactSet.io's growing user base to toggle their CI/CD workflows in a collaborative environment}
	\end{ressubsec}

	\begin{ressubsec}{Lead Software Engineer}{September 2015 - February 2017}{FactSet Research Systems}{Norwalk, CT}
		\ressubitem{Managed a team of three engineers, responsible for recruiting, team building, product development, project management, technical support, training, and technical architecture}
		\ressubitem{Assisted in FactSet's transition to a global company, adding region/zone support to FactSet.io, deploying hardware to five regional POPs allowing for self-service application deployment.  Built Apache Traffic Server CDN (cdn.factset.com), allowing for static asset deployment behind our Dyn-based anycast network.  Dramatically improved performance and reduced downtime for customers outside of the US}
		\ressubitem{Developed end-to-end application and infrastructure observability add-ons including healthchecks (\textit{Go}, \textit{Kafka}, \textit{Nagios} plugins), metrics (\textit{Go}, \textit{Kafka}, \textit{OpenTSDB}, \textit{MapR}, \textit{Grafana}), logging (\textit{Ruby on Rails}, \textit{Go}, \textit{Syslog}), alerting (\textit{Go}, \textit{Twilio}, \textit{Pushover}), and status pages (\textit{Ruby on Rails}) for internal incident communication}
		\ressubitem{Developed containerized Database as a Service platform for \textit{PostgreSQL}, \textit{Redis}, and \textit{RabbitMQ}, leveraging FactSet.io Container Manager, including automatic provisioning, monitoring, self-service backups and restores, and supporting advanced forking/follow/rollback features (\textit{Go}, \textit{WAL-E})}
		\ressubitem{Developed Identity API add-on, providing an authorization and authentication add-on for FactSet.io apps using \textit{OAuth}, \textit{OpenID Connect}, and \textit{JWT}, federating employee, service, and client accounts}
		\ressubitem{Introduced, deployed, and supported OpenStack Infrastructure as a Service, managing onboarding, training, day-to-day operational support, release upgrades, and high availability}
		\ressubitem{Developed \textit{Kerberos} single sign-on (SSO) authorization add-on for the FactSet.io (PaaS) HTTP Router}
		\ressubitem{Developed application idle support in FactSet.io, reducing hardware utilization}
		\ressubitem{Integrated \textit{Infoblox} DNS appliance automation and tooling into the build system, eliminating manual tickets and reducing turnaround time}
	\end{ressubsec}

	\pagebreak

	\begin{ressubsec}{Lead Unix Systems Engineer}{September 2013 - September 2015}{FactSet Research Systems}{Norwalk, CT}
		\ressubitem{Designed and built FactSet.io, a Platform as a Service, including Container Engine (\textit{Go}, \textit{Docker}), Container Manager (\textit{Go}, \textit{Zookeeper}, \textit{Google Omega}), HTTP Router (\textit{Node.js}, \textit{Redis Sentinel}), Git SSH Proxy (\textit{Go}), Slug Compiler (\textit{Ruby}, \textit{Cloud Foundry} and \textit{Heroku} buildpacks), Log Aggregation (\textit{Heroku Logplex}), Platform API (\textit{Go}, \textit{PostgreSQL}), Web Dashboard (\textit{Node.js}, \textit{React}, \textit{Webpack}), and CLI (\textit{Go}).  Used by 3500 applications and 575 active developers, driving FactSet's migration off OpenVMS monolithic codebase to microservices, allowing rapid integration of new company acquisitions, and primary infrastructure for FactSet's new web-based Workstation}
		\ressubitem{Led R\&D effort to find a new storage system for 20TB of client portfolio data on VMS, ultimiately recommending \textit{MapR Tables}, building prototype tools, benchmarking performance, and testing various failure scenarios}
		\ressubitem{Worked with various teams to graph on-call alerts, categorizing them, then reducing alerts dramatically with targeted improvements}
		\ressubitem{Improved system crash response with automated ticket creation, detailed system information (charts, last minute process list, related ticket list), and automatic recovery attempts, all but eliminiating the largest source of on-call alerts}
	\end{ressubsec}


	\begin{ressubsec}{Senior Unix Systems Engineer}{October 2011 - September 2013}{FactSet Research Systems}{Norwalk, CT}
		\ressubitem{Wrote a Java-based virtual serial port concentrator for VMWare to complement our \textit{Conserver} infrastructure for physical serial consoles}
		\ressubitem{Implemented inventory collection of hardware and software health, versions, and applications, integrating internal and third party APIs}
		\ressubitem{Introduced \textit{OpenStack Swift} object storage, drastically reducing storage costs for our commercial news and research content}
		\ressubitem{Led several large migration projects (HP G1 $\rightarrow$ G7, Solaris decommission in favor of Linux, Physical $\rightarrow$ Virtual), including assisting developers in application migration, providing a point of communiation, and organizing gantt charts and tickets}
		\ressubitem{Inherited the MySQL Database team, responsible for maintaining, upgrading, troubleshooting, and on-call support for 100+ content and application databases for various client-facing development teams}
		\ressubitem{Built network switchport automation to automatically change VLAN and update switchport descriptions during box build}
		\ressubitem{Designed and built a self-service \textit{MySQL} provisioning system based on \textit{Linux CGroups}, \textit{Pacemaker}, and \textit{Percona Monitoring} tools, with a Rails-based interface for management, requests, query review, metrics, and alerting}
		\ressubitem{Designed and built a self-service \textit{Elasticsearch} provisioning system for FactSet's new FactSearch search/autocomplete system and Central Logging Platform}
	\end{ressubsec}

	\begin{ressubsec}{Unix Systems Engineer}{June 2008 - October 2011}{FactSet Research Systems}{Norwalk, CT}
		\ressubitem{Maintained and supported 1000+ Red Hat Enterprise Linux and Oracle Solaris servers}
		\ressubitem{Led architecture and creation of server build and installation automation leveraging \textit{DNS}, \textit{PXELinux}, \textit{Kickstart}, \textit{Ruby on Rails}, and tight integration with internal ticketing system and monitoring systems.}
		\ressubitem{Developed multi-layer rolling patching system: daily \textit{Ksplice} live kernel-patching, weekly \textit{Yum} security package updates of non-sensitive packages, quarterly kernel and "minor release" reboot coordination with full Errata and RHN integration}
		\ressubitem{Planned and executed \textit{NetBackup} upgrades, \textit{Sun StorageTek L700} maintenance, and migration to \textit{Exagrid} disk storage}
		\ressubitem{Maintained several virtualization environments, starting on \textit{Xen}, migrating to \textit{KVM}, and eventually settling on \textit{VMWare}}
		\ressubitem{Managed provisioning, ACLs, quotas, and NFS exports on various file storage appliances (\textit{NetApp}, \textit{Isilon}, \textit{Oracle ZFS})}
		\ressubitem{Implemented configuration management while following change control policies and rolling installs (\textit{CFEngine}, \textit{Git}, \textit{Gerrit})}
	\end{ressubsec}

\pagebreak

	\begin{ressubsec}{Research Lab Staff}{May 2005 - May 2008}{ECE/CIS Department, University of Delaware}{Newark, DE}
		\ressubitem{Maintained and supported 700 Solaris, Windows, Linux, and Macintosh desktops and servers}
		\ressubitem{Built, configured, and maintained several undergraduate computer laboratories, research clusters, and servers for faculty and graduate students using \textit{UNIX shell scripting}, \textit{C programming}, \textit{MPICH}, and Linux/Solaris system administration skills}
		\ressubitem{Performed daily backups and quarterly archives using \textit{Bacula}, \textit{Amanda}, as well as custom Bash and batch scripts based on textit{rsync}}
		\ressubitem{Developed web-based projects such as a trouble ticket system, network monitoring application with \textit{mrtg}, backup information page, and software licence database using \textit{php}, \textit{python}, and \textit{mysql} with \textit{html}, \textit{css}, and \textit{ajax} using a \textit{subversion} code repository}
		\ressubitem{Developed console-based projects such as a \textit{procmail} and python based email parser and systems information database using \textit{RCS} revision control}
		\ressubitem{Installed and upgraded megabit and gigabit wiring in several campus buildings}
		\ressubitem{Developed wiki-based documentation site for documenting lab procedures and a Trac-based documentation site for software projects}
	\end{ressubsec}
\end{ressection}

\begin{ressection}{Education}
	\begin{ressubsec}{University of Delaware, Newark, DE}{May 2008}{Cumulative GPA 3.782/4.0}{}
		\ressubitem{B.E.E., Electrical Engineering (GPA 3.822)}
		\ressubitem{B.S., Computer Science with Honors (GPA 3.917)}
		\ressubitem{Minor: Mathematics}
	\end{ressubsec}
\end{ressection}

\begin{ressection}{Honors and Awards}
	\resitem{2015 Charles J. Snyder Technologist of the Year, FactSet}
\end{ressection}
\end{document}
